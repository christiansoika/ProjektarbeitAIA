\documentclass[a4paper,10pt]{article}
\usepackage[utf8]{inputenc}
\usepackage{tocloft}
\usepackage{nomencl}
\usepackage{pdfpages}
\usepackage{scrextend}
\usepackage{bm}
\usepackage{cite}
\usepackage{amsmath}

%bindet die benutzten Packages ein

\renewcommand*\vec[1]{\boldsymbol{#1}}
\renewcommand*\matrix[1]{\boldsymbol{#1}}

%setzt die fettgedruckte Schreibweise fuer Vektoren und Matrizen

%opening
\title{Investigations on one-way coupling effects of particle-laden decaying isotropic turbulent flows}
\author{Julian Stemmermann, Steffen Trienekens, Christian Soika}
\date{Aachen, 3rd of July 2017}

%schreibt die generellen Informationen aus

\numberwithin{equation}{section} %bestimmt die Nummerierung der Gleichungen, in diesem Fall nach Kapitel anstatt sie einfach durchzunummerieren

\makenomenclature %erstellt und druckt die Nomenklatur-Tabelle

\renewcommand{\refname}{}
\renewcommand{\nomname}{}

\begin{document}

\maketitle

\pagebreak

\tableofcontents{} %erstellt die Inhaltsangabe
 
\pagebreak

\section{Nomenclature}
\printnomenclature
\pagebreak

\section{Introduction}
in Computational Methods for Multiphase Flow ist auf den Seiten 3-9 ein interessantes Beispiel.
\pagebreak
\section{Mathematical models}
Sollen wir hier noch isotrope Turbulenz etc. erklaeren?
\subsection{Single-phase flow} %Christian
In this section the mathematical basics for understanding and simulating turbulent flows are discussed. However, it should be pointed out that this is no
complete treatise of the mathematical and physical basics. The reader can achieve further insight on this topic by looking at different books and papers, 
e.g. \cite{turbulentFlows}.
\newline
\subsubsection{The Navier-Stokes equations}
The Navier-Stokes-Equations are of great importance for understanding turbulent phenomena. This set of equations exists in forms for compressible and
incompressible fluids. For an infinitesimal small volume element $ \mathrm{d} \tau $ and using the cartesian coordinate system, 
they can be written in the so-called 'divergence form':
\begin{equation}
  \frac{\partial{\vec{Q}}}{\partial{t}} + \vec\nabla \vec{H} = 0
\end{equation}
\nomenclature{$\vec\nabla$}{Nabla-operator} %Erstellt automatisch die Nomenclature-Liste
\nomenclature{$\vec\Q$}{Container for conserved variables in the Navier-Stokes equations}
\nomenclature{$\vec{H}$}{Container for fluctuating variables in the Navier-Stokes equations}
It should be noted by the reader that this work only contains investigations about chemically inert fluids and particles, and that the simulation results
only fit under this condition. The vector $ \vec{Q} $ contains all the variables which are conserved, i.e. the density $ \rho $, 
the velocity $ \vec{u} $ and the specific inner energy $ E $: 
\nomenclature{$\rho$}{Density}
\nomenclature{$\vec\u$}{Three-dimensional velocity}
\nomenclature{$E$}{Specific inner energy}
\begin{equation}
 \vec{Q}= \left( \begin{array}{c}\rho\\\rho \vec{u}\\\rho E \end{array} \right)
\end{equation}
$ \vec{H} $ is the flux vector which stores all the floating variables and may be split up into two parts: 
\begin{equation}
 \vec{H} = \vec{H^\mathrm{i}} + \vec{H^\mathrm{v}}
\end{equation}
\nomenclature{$\vec{H^\mathrm{i}}$}{Stores the inviscid variables in the flux-vector included in the Navier-Stokes equations}
\nomenclature{$\vec{H^\mathrm{v}}$}{Stores the viscous variables in the flux-vector included in the Navier-Stokes equations}
The contents of the two vectors are displayed below:
\begin{equation}
 \vec{H^\mathrm{i}} = \left( \begin{array}{c}\rho \vec{u}\\\rho \vec{u} \vec{u} + p\\\vec{u} (\rho E + p) \end{array} \right)
\end{equation}
\nomenclature{$p$}{Pressure}
\begin{equation}
 \vec{H^\mathrm{v}} = - \frac{1}{Re} \left( \begin{array}{c}\ 0 \\ \matrix{\tau}\\ \matrix{\tau} \vec{u} + \vec{q} \end{array} \right)
\end{equation}
\nomenclature{$Re$}{Reynolds number}
\nomenclature{$\matrix{\tau}$}{Stress tensor}
\nomenclature{$\vec{q}$}{Heat conduction}
$ \vec{H^\mathrm{i}} $ is called inviscid flux and contains only the variables that are independent of the fluids viscosity, it describes the way a fluid 
with zero viscosity would behave. In contrast, the viscous flux $ \vec{H^\mathrm{v}} $ represents the effects of viscosity. The Reynolds number 
$ Re = \frac{\rho v d}{\eta} $ is defined to be the ratio of inertia to tenacity, which makes it very valuable for understanding turbulent flows. This is also due to the 
fact that two familiar objects with the same Reynolds number behave similar in turbulence. One can assume that flows with $ Re << 1 $ are laminar and flows with 
$ Re >> 1 $ are turbulent.
To solve the Navier-Stokes-Equations, more information regarding some variables is required. For Calculating the specific inner Energy $ E $ 
and the heat conduction $ \vec{q} $, the following equations are used:
\begin{equation}
 E = e  \frac{1}{2} \vec{|u|}^2
\end{equation}
\nomenclature{$e$}{Specific internal energy}
\begin{equation}
 \vec{q} = - \frac{\mu}{Pr (\gamma - 1)} \vec\nabla T
\end{equation}
\nomenclature{$Pr$}{Prandtl number}
\nomenclature{$\mu$}{Dynamic viscosity}
with 
\begin{equation}
 \gamma = \frac{c_p}{c_v}
\end{equation}
\nomenclature{$c_p$}{Specific isobaric heat capacity}
\nomenclature{$c_v$}{Specific isochoric heat capacity}
and the Prandtl number
\begin{equation}
 Pr = \frac{\mu_\infty c_p}{k_t}
\end{equation}
\nomenclature{$k_t$}{thermal conductivity}
using the specific heat capacities of the fluid $ c_v $ and $ c_p $.
If one could assume that the fluid is a newtonian fluid, the linear correlation between stress and the rate of strain results in:
\begin{equation}
 \matrix{\tau} = 2 \mu \matrix{S} - \frac{2}{3} \mu (\vec\nabla * \vec{u}) \matrix{I}
\end{equation}
\nomenclature{$\matrix{I}$}{Identity tensor}
\nomenclature{$\matrix{S}$}{Rate-of-strain-tensor}
\nomenclature{$T$}{Temperature}
in which $ \matrix{S} = \frac{(\vec\nabla \vec{u})(\vec\nabla \vec{u})^T}{2} $ denotes the rate-of-strain-tensor. Additionally, the viscosity
$ \mu $ can be aprroximated through Sutherland's law, which is based on the ideal gas-theory: 
\begin{equation}
 \mu (T) = \mu_\infty (\frac{T}{t_\infty})^{3/2} \frac{T_\infty + S}{T + S}
\end{equation}
\nomenclature{$S$}{Sutherland temperature}
\nomenclature{$R$}{Universal gas constant}
S is in this case the Sutherland temperature.
To achieve closure the caloric state equation $ e = c_v T $ and the state eqaution for an ideal gas $
p = \rho R T $ are used. The specific gas constant is determined by $ R = c_p - c_v $. 
These equations form a set of partial differential equations, so for solving them starting values are needed. These are initialized at the first timestep of 
the simulation. To achieve physical solutions, 150 timesteps are computed before the particles are initialized, so the turbulence can evolve from the 
synthetic values to a natural flow field. 
\newline
Christoph Siewert:
-2.1 bis 2.6
Stephan Fritz:
-Navier-Stokes-Gleichungen (Anhang B)
Randbedingungen?
\pagebreak
\subsection{Particle dynamics} %Steffen
Siewert:
-3.1a-3.14 (spherical particles) OHNE GRAVITATION
Stokes Drag/Stokes Coefficient
Filterung (Fritz) ->Viskositaet durch numerischen Fehler, Smagorinksy nicht benutzen
\pagebreak
\section{Numerical methods} %Julian
To simulate flows like those described above we have two options. The direct numerical simulation (DNS) is the easier one to understand, although it is numerically very expensive. The Large-eddy simulation (LES) is numerically more capable, still we must accept certain inaccuracies. These two numerical methods are now discussed in the following chapter.
\subsection{Direct numerical simulation}
The basis of the direct numerical simulation (DNS) are the Navier-Stokes equations as described above. The idea is that the computer is very good at calculating and solves these equations completely. This provides us a very accurate result, as all scales of motion are being resolved. Still it requires an immense level of comptational resources which increases rapidly with the Reynolds number. These computational resources were not available until the 1970s \cite{turbulentFlows}.
\newline
\pagebreak
Pojektion (noComputationalParticles)
implizite LES (Motivation fuer LES -  Pope Chapter 9, Bild 9.4), DNS
\pagebreak
\section{Results}
%\begin{figure}
 %\centering %setzt das Bild mittig
 %\includegraphics{} %Dateiname einfuegen um Datei zu verwenden
 %\caption{} %setzt die Bildunterschrift
% \label{} %setzt das Label
%\end{figure}

Graphen (particleFree rot, Laden gruen)
\pagebreak
\section{Conclusion}
\pagebreak
\section{References}
\nocite{*} %ermoeglicht, dass auch Literatur, welche nicht zitiert wurde in der Bibliography auftaucht
\bibliography{Projektarbeit} %ruft die Bibliography-Datei auf
\bibliographystyle{plain} %setzt den Zitierstil
\pagebreak

\end{document}

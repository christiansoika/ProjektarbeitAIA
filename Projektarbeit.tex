\documentclass[a4paper,10pt]{article}


\usepackage[utf8]{inputenc}
\usepackage{tocloft}
\usepackage{nomencl}
\usepackage{pdfpages}
\usepackage{scrextend}
\usepackage{bm}
\usepackage{cite}
\usepackage{amsmath}

%bindet die benutzten Packages ein

\renewcommand*\vec[1]{\boldsymbol{#1}}
\renewcommand*\matrix[1]{\boldsymbol{#1}}

%setzt die fettgedruckte Schreibweise fuer Vektoren und Matrizen

%opening
\title{Investigations on one-way coupling effects of particle-laden decaying isotropic turbulent flows}
\author{Julian Stemmermann, Steffen Trienekens, Christian Soika}
\date{Aachen, 3rd of July 2017}

%schreibt die generellen Informationen aus

\numberwithin{equation}{section} %bestimmt die Nummerierung der Gleichungen, in diesem Fall nach Kapitel anstatt sie einfach durchzunummerieren

\makenomenclature %erstellt und druckt die Nomenklatur-Tabelle

\renewcommand{\refname}{}
\renewcommand{\nomname}{}

\begin{document}

\maketitle

\pagebreak

\tableofcontents{} %erstellt die Inhaltsangabe
 
\pagebreak

\section{Nomenclature}
\printnomenclature
\pagebreak

\section{Introduction}
in Computational Methods for Multiphase Flow ist auf den Seiten 3-9 ein interessantes Beispiel.
\pagebreak
\section{Mathematical models}
\pagebreak
\subsection{Single-phase flow} %Christian
In this section the mathematical basics for understanding and simulating turbulent flows are discussed. However, it should be pointed out that this is no
complete treatise of the mathematical and physical basics. The reader can achieve further insight on this topic by looking at different books and papers, 
e.g. \cite{turbulentFlows}.
\newline
\subsubsection{The Navier-Stokes equations}
The Navier-Stokes-Equations are of great importance for understanding turbulent phenomena. This set of equations exists in forms for compressible and
incompressible fluids. For an infinitesimal small volume element $ \mathrm{d} \tau $ and using the cartesian coordinate system, 
they can be written in the so-called 'divergence form':
\begin{equation}
  \frac{\partial{\vec{Q}}}{\partial{t}} + \vec\nabla \vec{H} = 0
\end{equation}
\nomenclature{$\vec\nabla$}{Nabla-operator} %Erstellt automatisch die Nomenclature-Liste
\nomenclature{$\vec\Q$}{Container for conserved variables in the Navier-Stokes equations}
\nomenclature{$\vec{H}$}{Container for fluctuating variables in the Navier-Stokes equations}
It should be noted by the reader that this work only contains investigations about chemically inert fluids and particles, and that the simulation results
only fit under this condition. The vector $ \vec{Q} $ contains all the variables which are conserved, i.e. the density $ \rho_\mathrm{f} $, 
the velocity $ \vec{u} $ and the specific inner energy $ E $: 
\nomenclature{$\rho_\mathrm{f}$}{Fluid density}
\nomenclature{$\vec\u$}{Three-dimensional velocity}
\nomenclature{$E$}{Specific inner energy}
\begin{equation}
 \vec{Q}= \left( \begin{array}{c}\rho_\mathrm{f}\\\rho_\mathrm{f} \vec{u}\\\rho_\mathrm{f} E \end{array} \right)
\end{equation}
$ \vec{H} $ is the flux vector which stores all the floating variables and may be split up into two parts: 
\begin{equation}
 \vec{H} = \vec{H^\mathrm{i}} + \vec{H^\mathrm{v}}
\end{equation}
\nomenclature{$\vec{H^\mathrm{i}}$}{Stores the inviscid variables in the flux-vector included in the Navier-Stokes equations}
\nomenclature{$\vec{H^\mathrm{v}}$}{Stores the viscous variables in the flux-vector included in the Navier-Stokes equations}
The contents of the two vectors are displayed below:
\begin{equation}
 \vec{H^\mathrm{i}} = \left( \begin{array}{c}\rho_\mathrm{f} \vec{u}\\\rho_\mathrm{f} \vec{u} \vec{u} + p\\\vec{u} (\rho_\mathrm{f} E + p) \end{array} \right)
\end{equation}
\nomenclature{$p$}{Pressure}
\begin{equation}
 \vec{H^\mathrm{v}} = - \frac{1}{Re} \left( \begin{array}{c}\ 0 \\ \matrix{\tau}\\ \matrix{\tau} \vec{u} + \vec{q} \end{array} \right)
\end{equation}
\nomenclature{$Re$}{Reynolds number}
\nomenclature{$\matrix{\tau}$}{Stress tensor}
\nomenclature{$\vec{q}$}{Heat conduction}
$ \vec{H^\mathrm{i}} $ is called inviscid flux and contains only the variables that are independent of the fluids viscosity, it describes the way a fluid 
with zero viscosity would behave. In contrast, the viscous flux $ \vec{H^\mathrm{v}} $ represents the effects of viscosity. The Reynolds number 
$ Re = \frac{\rho_\mathrm{f} v d}{\eta} $ is defined to be the ratio of inertia to tenacity, which makes it very valuable for understanding turbulent flows. This is also due to the 
fact that two familiar objects with the same Reynolds number behave similar in turbulence. One can assume that flows with $ Re << 1 $ are laminar and flows with 
$ Re >> 1 $ are turbulent.
To solve the Navier-Stokes-Equations, more information regarding some variables is required. For Calculating the specific inner Energy $ E $ 
and the heat conduction $ \vec{q} $, the following equations are used:
\begin{equation}
 E = e  \frac{1}{2} \vec{|u|}^2
\end{equation}
\nomenclature{$e$}{Specific internal energy}
\begin{equation}
 \vec{q} = - \frac{\mu}{Pr (\gamma - 1)} \vec\nabla T
\end{equation}
\nomenclature{$Pr$}{Prandtl number}
\nomenclature{$\mu$}{Dynamic viscosity}
with 
\begin{equation}
 \gamma = \frac{c_p}{c_v}
\end{equation}
\nomenclature{$c_p$}{Specific isobaric heat capacity}
\nomenclature{$c_v$}{Specific isochoric heat capacity}
and the Prandtl number
\begin{equation}
 Pr = \frac{\mu_\infty c_p}{k_t}
\end{equation}
\nomenclature{$k_t$}{thermal conductivity}
using the specific heat capacities of the fluid $ c_v $ and $ c_p $.
If one could assume that the fluid is a newtonian fluid, the linear correlation between stress and the rate of strain results in:
\begin{equation}
 \matrix{\tau} = 2 \mu \matrix{S} - \frac{2}{3} \mu (\vec\nabla * \vec{u}) \matrix{I}
\end{equation}
\nomenclature{$\matrix{I}$}{Identity tensor}
\nomenclature{$\matrix{S}$}{Rate-of-strain-tensor}
\nomenclature{$T$}{Temperature}
in which $ \matrix{S} = \frac{(\vec\nabla \vec{u})(\vec\nabla \vec{u})^T}{2} $ denotes the rate-of-strain-tensor. Additionally, the viscosity
$ \mu $ can be aprroximated through Sutherland's law, which is based on the ideal gas-theory: 
\begin{equation}
 \mu (T) = \mu_\infty (\frac{T}{t_\infty})^{3/2} \frac{T_\infty + S}{T + S}
\end{equation}
\nomenclature{$S$}{Sutherland temperature}
\nomenclature{$R$}{Universal gas constant}
S is in this case the Sutherland temperature.
To achieve closure the caloric state equation $ e = c_v T $ and the state eqaution for an ideal gas $
p = \rho R T $ are used. The specific gas constant is determined by $ R = c_p - c_v $. 
These equations form a set of partial differential equations, so for solving them starting values are needed. These are initialized at the first timestep of 
the simulation. To achieve physical solutions, 150 timesteps are computed before the particles are initialized, so the turbulence can evolve from the 
synthetic values to a natural flow field. 
\pagebreak
\subsection{Particle dynamics} %Steffen
Siewert:
-3.1a-3.14 (spherical particles) OHNE GRAVITATION
Stokes Drag/Stokes Coefficient
Filterung (Fritz) ->Viskositaet durch numerischen Fehler, Smagorinksy nicht benutzen
\pagebreak

To get a usefuell equation of motion for the particles in the flow, we use the Euler Lagrangian approach, as it is common in Direct Numerical Simulations (DNS) and Large Eddy Simulations (LES).
In this work we deal with small and heavy particles, that have a spherical shape. Their radius $ r_p $ is even smaller than the Kolmogorov scale $ \eta $, but also large enough to neglect the Brownian motion. The density of the particles $ \rho_\mathrm{p} $ is much higher than that of the fluid $ \rho_\mathrm{f} $ . 
In addition due to the very low particle concentration we can neglect the influence of the particles on the mean flow and on each other. This simplification is called one-way coupling. That means that the cross each other without any effect. That means particle collision is neglected.
After all this simplifications we obtain a simplified version of the Maxey-Riley equations. 
\nomenclature{$r_p$}{Particle radius}
\nomenclature{$\eta$}{Kolomogorov scale}
\nomenclature{$\rho_\mathrm{p}$}{Particle density}
\begin{equation}
 \frac{\partial{\vec{x_\mathrm{p}}}}{\partial{t}} = \vec{v_\mathrm{p}}
\end{equation}
\nomenclature{$\vec{x_\mathrm{p}}{Particle position}
\nomenclature{$\vec{v_\mathrm{p}}{Particle velocity}
\begin{equation}
 \frac{\partial{\vec{v_\mathrm{p}}}}{\partial{t}} = \frac{f_mathrm{D}}{tau_mathrm{p}} (\vec\u(x_mathrm{p}) - \vec\v_mathrm{p})
 \end{equation}
 \nomenclature{$f_mathrm{D}$}{Drag correction}
 \nomenclature{$\tau_mathrm{p}$}{Particle response time}
 
 It shows that the particle motion is only depending on the the hydrodynamical drag force which results out of the actual difference between the fluid velocity \var\u(x_mathrm{p}) at particle position \var\x_p and the particle velocity \var\v_p.
 \tau_mathrm{p} is the particle response time and a factor to obtain the drag force in Stokes flow conditions.
 To take the case of an Reynolds number \Re of order 1 into account the correction factor f_D = 1 + 0.15 Re_p^(0.687) is used.
 
 The biggest simplification of this Lagrangian approach is that the interaction of particles coming close together is neglected. 

\section{Numerical methods} %Julian
To simulate flows like those described above we have two options. The direct numerical simulation (DNS) is the easier one to understand, although it is numerically very expensive. The Large-eddy simulation (LES) is numerically more capable, still we must accept certain inaccuracies. These two numerical methods are now discussed in the following chapter.
\subsection{Direct numerical simulation}
The basis of the direct numerical simulation (DNS) are the Navier-Stokes equations as described above. The idea is that the computer is very good at calculating and solves these equations completely. This provides us a very accurate result, as all scales of motion are being resolved. Still it requires an immense level of comptational resources which increases rapidly with the Reynolds number. These computational resources were not available until the 1970s \cite{turbulentFlows}.
\newline
\pagebreak
Pojektion (noComputationalParticles)
implizite LES (Motivation fuer LES -  Pope Chapter 9, Bild 9.4), DNS
\pagebreak
\section{Results}
\subsection{Boundary conditions and simulation properties}
The simulations where carried out using ZFS, the simulation tool developed and implemented at the Institute of Aerodynamics at RWTH Aachen University 
\cite{anAdaptiveMultilevelMultigridFormulationForCartesianHierarchicalGridMethods} \cite{aStrictlyConservativeCartesianCutCellMethodForCompressibleViscousFlowsOnAdaptiveGrids}. 
The tool is capable of simulating finite-volume flows of compressible fluids. 
In this case the turbulence was simulated on a cubic grid using $64^3$,$96^3$,$128^3$ and $256^3$. The first three cases where simulated using LES, 
the case in which $256^3$ cells where used is carried out as DNS. Futher information can be gained by looking at \cite[p.344-357 for DNS and p. 558-639 for LES]{turbulentFlows}.
For simplification, the special case of isotropic turbulence was used. For this idealised flow form the statistical 
velocities are invariant in all directions of the grid. It follows that the flows velocity are invariant for rotations and reflections. 
The turbulence was initialised using a seed-based random generator. To achieve physical results, the simulation was carried out  to timestep 150, 
at which a restart file was written out. This procedure insures a fully developed turbulent 
flow, which has emancipated from the initialisation. In this flow field, a specific number of spherical particles where injected. 
The velocties of the fluid where interpolated for each particle to match the velocities of particles and fluid as accurately as possible.

boundary conditions, for example Reynolds number?

%\begin{figure}
 %\centering %setzt das Bild mittig
 %\includegraphics{} %Dateiname einfuegen um Datei zu verwenden
 %\caption{} %setzt die Bildunterschrift
% \label{} %setzt das Label
%\end{figure}

Graphen (particleFree rot, Laden gruen)
\pagebreak
\section{Conclusion}
\pagebreak
\section{References}
\nocite{*} %ermoeglicht, dass auch Literatur, welche nicht zitiert wurde in der Bibliography auftaucht
\bibliography{Projektarbeit} %ruft die Bibliography-Datei auf
\bibliographystyle{plain} %setzt den Zitierstil
\pagebreak

\end{document}
